%%%%%%%%%%%%%%%%%%%%%%%%%%%%%%%%%%%%%%%%%%%%%%%%%%%%%%%%%%%%%%%%%%%%%%%%%%%%%%%
%%                                                                 INTRODUCTION
%%%%%%%%%%%%%%%%%%%%%%%%%%%%%%%%%%%%%%%%%%%%%%%%%%%%%%%%%%%%%%%%%%%%%%%%%%%%%%%
%%                                the introductory and motivating first chapter



%______________________________________________________________________________
%                                                     Einführung und Motivation
\chapter{Einführung und Motivation}

\begin{quote}
    So sähe beispielsweise ein \textit{abstract} für eine separates Kapitel aus. Hier müsste relativ viel stehen, damit man
    auch sieht, wie mit Zeilenumbrüchen agiert wird.
\end{quote}


\section{Testsection}
Hier kommt das aller erste einführende Kapitel hin \cite{fitzgerald:realigning_research_and_practice}.
Hier wird zum ersten mal ein \textbf{Akronym} benutzt, \ac{TRT}. Nun wird das noch ein zweites mal benutzt, \ac{TRT}.

\section{Another TestSection}
Lorem ipsum und so weiter und so fort
Noch mehr Lückenfüller-Text
