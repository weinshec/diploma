%%%%%%%%%%%%%%%%%%%%%%%%%%%%%%%%%%%%%%%%%%%%%%%%%%%%%%%%%%%%%%%%%%%%%%%%%%%%%%%
%%                                                                 INTRODUCTION
%%%%%%%%%%%%%%%%%%%%%%%%%%%%%%%%%%%%%%%%%%%%%%%%%%%%%%%%%%%%%%%%%%%%%%%%%%%%%%%
%%                                the introductory and motivating first chapter



%______________________________________________________________________________
%                                                     Einführung und Motivation
\chapter{Einführung und Motivation}


Das Standardmodell der Teilchenphysik ist äußerst erfolgreich in der
Beschreibung des Aufbaus der bekannten Materie aus fundamentalen Bausteinen und
Wechselwirkungen. Dabei konnte die Existenz einiger Teilchen bereits vor deren
experimentellen Nachweis vorhergesagt werden. Der jüngeste Erfolg, die
Entdeckung des Higgs-Bosons am \ac{LHC}, komplettiert die Liste der
vorhergesagten Teilchen.

Das Standardmodell bezieht in der Erklärung der fundamentalen Wechselwirkungen
zwischen den Bausteinen drei der vier bekannten Grundkräfte ein, die starke,
die schwache und die elektromagnetische Kraft. Die beiden letztgenannten vermag
das Standardmodell dabei in einer einzigen elektroschwachen Theorie
zusammenzufassen. Im Zuge dieser Vereinheitlichung wird der sogenannte schwache
Mischungswinkel $\sin^2\theta_W$ eingeführt, der die Mischung der fundamentalen
Eichbosonen der elektroschwachen Theorie zu den beobachtbaren Austauschbosonen
($Z,\gamma$) parametrisiert. Der Wert des schwachen Mischungswinkels wird dabei
von der Theorie nicht festgelegt und bleibt als freier Parameter durch das
Expriment zu bestimmen. Die paritätsverletzende Struktur der schwachen
Wechselwirkung induziert im betrachten Streuprozess dieser Arbeit eine
Asymmetrie in der Emissionsrichtung der Produktteilchen, die einen Zugang zur
Bestimmung des schwachen Mischungswinkels liefert. In vorangegangenen
Experimenten konnte in Elektron-Positron Kollisionen bei
\acs{LEP}\footnote{\acf{LEP}} und \acs{SLC}\footnote{\acf{SLC}} der
Mischungswinkel bereits mit hoher Präzision gemessen werden. Mit dem \ac{LHC}
steht nun auch ein Hadron-Collider diesem Zweck zur Verfügung. Die Messung der
Vorwärts-Rückwärts und die daran anschließende Extraktion des schwachen
Mischungswinkels in Proton-Proton Kollisionen am \ac{LHC} mit dem
ATLAS-Detektor sind das Ziel dieser Arbeit.

Im ersten Kapitel wird dafür zunächst der theoretische Grundstein gelegt. Es
wird dabei das Standardmodell mit seinen Bausteinen und Wechselwirkungen
erläutert, wobei ein besonderes Augenmerk auf der elektroschwachen Theorie
liegt. Anschließend wird die Entstehung der Vorwärts-Rückwärts Asymmetrie
erläutert und wichtige Aspekte im Hinblick auf deren Messung in Proton-Proton
Kollisionen diskutiert. Das darauffolgende Kapitel \ref{experimenteller_aufbau}
widmet sich dann der Beschreibung des experiementellen Werkzeuges. Nach einer
kurzen Vorstellung des \acf{LHC} wird der ATLAS-Detektor im nötigen Detail
beschrieben, sowie die Funktionsweise der relevanten Komponenten erläutert.
Die Identifikation und Rekonstruktion von Elektronen in ATLAS ist Thema gegen
Ende dieses Kapitel. Danach schließt sich ein kurzes Kapitel über den
verwendeten Datensatz, die Erzeugung von Monte-Carlo Simulationen und der
Erstellung von Selektionskriterien zur Diskriminierung des Untergrundes an.
Kapitel \ref{energy_calibration} beschäftigt sich dann mit der für die
Präzision der angestrebten Analyse wichtigen Kalibration der Elektronen bei
hohen Pseudorapiditäten. Hierbei wird folgt nach einer eher theoretischen
Vorstellung der Kalibrationsmethode die eigentliche Bestimmung der
Kalibrationsfaktoren. Das letzte Kapitel \ref{afb} beginnt zuerst mit der
Charakterisierung des analysierten Datensatzes, auf die die simulations- und
datengestützte Abschätzung des Untergrundes folgt. Die Verteilung des
Elektron\-emissionswinkels, und die daraus resultierende Asymmetrie sind dann
Gegenstand der nachfolgenden Abschnitte. Nach der anschließenden Beschreibung
der Extraktionsmethode des schwachen Mischungswinkels, folgt dessen Bestimmung
aus der Verteilung der Vorwärts-Rückwärts Asymmetrie. Das Kapitel schließt mit
einer kritischen Diskussion und dem Ausblick auf mögliche Verbesserungen, sowie
dem Vergleich mit den Ergebnissen vorheriger Experimente.

