%%%%%%%%%%%%%%%%%%%%%%%%%%%%%%%%%%%%%%%%%%%%%%%%%%%%%%%%%%%%%%%%%%%%%%%%%%%%%%%
%%                                                                      SUMMARY
%%%%%%%%%%%%%%%%%%%%%%%%%%%%%%%%%%%%%%%%%%%%%%%%%%%%%%%%%%%%%%%%%%%%%%%%%%%%%%%
%%                                                 summary  of the whole thesis


\chapter{Zusammenfassung und Ausblick}
\label{summary}

Das Ziel dieser Arbeit war es, die Vorwärts-Rückwärts Asymmetrie in
Proton-Proton Kollisionen bei Impulsüberträgen nahe der $Z$-Masse zu messen und
anschließend den effektiven schwachen Mischungswinkel
$\sin^2\theta_W^\text{eff,l}$ zu extrahieren.

Mit den 2012 aufgenommen Daten des ATLAS-Detektors am LHC bei $\sqrt{s}=8\TeV$
Schwerpunktsenergie wurde der Drell-Yan Prozess mit Elektron-Positron Paaren im
Endzustand betrachtet ($pp\rightarrow \gamma*/Z +X \rightarrow e^+e^-+X$). Der
Datensatz umfasste dabei insgesammt eine integrierte Luminosität von
$20.3\fb^{-1}$. Da die Messung der Vorwärts-Rückwärts Asymmetrie in
Proton-Proton Kollisionen unabdingbar von der Einbeziehung von
Elektronen/Positronen mit hohen Pseudorapiditäten profitiert, wurde ein Satz
von Kalibrationsfaktoren bestimmt, der die Messung der Energie in diesem
erweiterten Akzeptanz-Bereich korrigiert. Für die Asymmetrie wurden die
relevanten Elektron-Positron Paare selektiert und deren kinematischen
Eigenschaften mit Monte-Carlo Simulationen und datengestützten
Untegrundabschätzungen verglichen. Aus der Verteilung des Emissionswinkels der
Elektronen relativ zur Richtung des einfallenden Quarks wurde in einem Bereich
der invarianten Masse zwischen $75\GeV$ und $105\GeV$ die Asymmetrie dieser
Winkelverteilung berechnet und ebenfalls mit der Vorhersage aus Simulation und
Untergrund verglichen. Die Extraktion des schwachen Mischungswinkels aus den
Verteilungen der Vorwärts-Rückwärts Asymmetrie wurde mithilfe simulierter
Vorlagen  und Anpassungstests durchgeführt. Dabei konnte ein Wert von 
\begin{equation*}
    \sin^2\theta_W^\text{eff,l} \;=\; 0.2314
        \;\pm\; 0.0005 \;\text{(stat)}
\end{equation*}
ermittelt werden, der sowohl mit vorangegangen Messungen in Hadron-Hadron
Kollisionen vergleichbar ist, als auch mit dem Weltmittelwert innerhalb der
Toleranzen übereinstimmt. Die Einbeziehung systematischer Effekte auf diesen
Wert konnte in dieser Arbeit nicht mehr durchgeführt werden.

In den kommenden Jahren wird ATLAS immer größere Datenmengen sammeln, sodass
eine weitere Reduktion der bestehenden statistischen Unsicherheiten ermöglicht
wird.  Desweiteren stehen mit dem Abschluss der derzeit stattfinden Ausbauphase
des \ac{LHC} dann Proton-Proton Kollisionen mit bis zu $14\TeV$
Schwerpunktsenergie zur Verfügung, die für Präzisionstests des Standard-Modells
und der Suche nach neuer Physik, auch gerade in der Vorwärts-Rückwärts
Asymmetrie gereichen können. 
