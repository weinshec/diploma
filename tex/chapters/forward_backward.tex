%%%%%%%%%%%%%%%%%%%%%%%%%%%%%%%%%%%%%%%%%%%%%%%%%%%%%%%%%%%%%%%%%%%%%%%%%%%%%%%
%%                                                   FORWARD BACKWARD ASYMMETRY
%%%%%%%%%%%%%%%%%%%%%%%%%%%%%%%%%%%%%%%%%%%%%%%%%%%%%%%%%%%%%%%%%%%%%%%%%%%%%%%
%%                       the chapter about the AFB and the sin2theta extraction



%______________________________________________________________________________
%                     Messung der Asymmetrie und des schwachen Mischungswinkels
\chapter{Messung der Asymmetrie und des schwachen Mischungswinkels}
\label{afb}

\begin{quote}
    The abstact comes last
\end{quote}



%______________________________________________________________________________
%                                                      Datensätze und Selektion
\section{Datensätze und Selektion}
\label{afb:selection}

% + benutzter Datensatz
% + Selektion / Akzeptanz
% + Benutzte Gewichte

% + Kontroll-Plots der Masse / Winkelverteilung / Phi

Die Messung der Vorwärts-Rückwärts Asymmetrie und die anschließende Extraktion
des schwachen Mischungswinkels basiert, wie schon die vorangegangene
Energiekalibration, auf dem vollen Datensatz des Jahres 2012 mit $8\TeV$
Schwerpunktsenergie und einer integrierten Luminosität\footnote{nach Anwendung
der \ac{GRL}} von $20.3\fb^{-1}$. Zur Selektion der interessanten Ereignisse
werden die in Abschnitt \ref{data_sim_selection:selection} beschriebenen
Schnitte angewendet (siehe Tabelle \ref{tab:data_selection}), wobei die
Unterteilung der Ereignismenge nach solchen mit Beteiligung von
Vorwärts-Elektronen (CF) und reinen Zentral-Zentral Ereignissen (CC) zunächst
beibehalten wird. Die Analyse und Messung wird dann in diesen beiden Kanälen
zuerst getrennt durchgeführt und abschließend zu einem kombiniertem Ergebnis
zusammengefasst.
\begin{table} [h]
    \centering
    \begin{tabular}{|l|r|r|}
        \hline
        & \multicolumn{2}{|c|}{Elektronkandidaten}   \\
        \textbf{Schnitt} & \textbf{CC} & \textbf{CF} \\  % escale CF
        \hline\hline
        \ac{GRL}           & 389741202 & 389741202   \\  % 3954702112
        Detektor Status    & 389740981 & 389740981   \\  % 3945153030
        Trigger            & 158129884 & 158129884   \\  % 1402546381
        primärer Vertex    & 157901943 & 157901943   \\  % 1400899527
        Pseudorapidität    & 149728397 & 122396071   \\  % 1328016208
        Transversal-Impuls &  63811039 &   8342860   \\  %  498956943
        Autor              &  17186000 &   7841721   \\  %  378463236
        ID                 &   4326530 &    995073   \\  %   76603185
        Ladung             &   4266425 &  (995073)   \\  %           
        OQ                 &   4245028 &    988991   \\  %   75178238
        \hline
    \end{tabular}
    \caption[Anzahl der Elektronkandidaten in Daten für CC und CF Selektion]
        {Anzahl der Elektronkandidaten in Daten nach für CC und CF Selektion.
        Die Auflistung ist inklusiv, d.h. untere Schnitte enthalten implizit
        alle vorangegangenen Schnitte. \development (check numbers with
        calibration)}
    \label{tab:data_selection}
\end{table}
Auf alle Elektronkandidaten im Datensatz werden zuvor noch die entsprechenden
Kalibrationfaktoren für die Energie angewendet (siehe Kapitel
\ref{data_sim_selection:data} und \ref{energy_calibration}), sodass die
Selektion auf kalibrierten Energiewerten stattfinden kann.

Zur simulationsseitigen Beschreibung des Signalprozesses
$pp \rightarrow \gamma^*/Z(e^+e^-) + X$, also dem elektroschwachen Zerfall
eines $Z$-Bosons bzw. virtuellen Photons in ein Elektronpaar\footnote{auch hier
wird der Begriff \textit{Elektron} wieder synonym für Teilchen und Antiteilchen
verwendet}, wird das von \textsc{Powheg+Pythia} generierte Drell-Yan
Monte-Carlo verwendet (siehe Kapitel \ref{used_mc_samples}). Es werden
identische Selektionkriterien, wie auf den Daten angewendet, wobei die Energie
der Elektronkandidaten zuvor einer Auflösungskorrektur unterzogen wird. Dabei
wird eine künstliche Verschmierung der Energie eingeführt, um die oftmals zu
optimistisch angenommene Energieauflösung im Monte-Carlo nachträglich zu
korrigieren\footnote{weitere Details siehe Kapitel \ref{energy_calibration}}.
Zudem wird für jedes selektierte Ereigniss ein spezifisches Gewicht errechnet,
mit dem dieses Ereignis bei der Erstellung von statistischen Verteilungen
beträgt. Es gehen hierbei Ereignis spezifische, sowie Elektron spezifische
Korrekturfaktoren in die Bildung dieses Gewichtes ein, deren Ursprung und
Benutzung in Kapitel \ref{mc_corrections} motiviert wurde. Zusammengefasst
ergeben sich die Gesamtgewichte zu
\begin{align}
    w_\text{event}^\text{CC} &= w_\text{pu} \cdot w_\text{kFactor} \cdot
        w_\text{vtx} \cdot w_\text{gen} \cdot \prod_{i=1,2} w_{\text{ID},i}
        \cdot w_{\text{spur},i} \cdot w_{\text{trigger},i}
        \\[2pt]
    w_\text{event}^\text{CF} &= w_\text{pu} \cdot w_\text{kFactor} \cdot
        w_\text{vtx} \cdot w_\text{gen} \cdot w_{\text{ID},1} \cdot
        w_{\text{ID},2} \cdot w_{\text{spur}} \cdot w_{\text{trigger}}
\end{align}
wobei der Index $i$ in \ac{CC} Ereignissen verdeutlicht, dass die
entsprechenden Korrekturfaktoren für beide Zentral-Elektronen separat eingehen.
In \ac{CF} Ereignissen findet diese Unterscheidung nur für den Faktor der
Identifikationseffizienz statt, da die beiden übrigen Effizienzkorrekturen
nicht für Vorwärts-Elektronen definiert sind.

Zur Abschätzung der Beiträge von Untergrundprozessen wird eine Kombination aus
Simulation und datengestützter Betimmung verwendet. Beide Aspekte werden im
Folgenden näher erläutert und deren Anteil an den selektierten Ereignissen
quantifiziert.



\subsection{Simulierte Untergrundbeiträge}
\label{afb:monte_carlos}

% + Benutzte Monte-Carlos

Neben dem Drell-Yan Prozess mit zwei Elektronen im Endzustand gibt es
zahlreiche weitere Prozesse, die eine ähnliche Signatur im Detektor zeigen und
deshalb von der angewendeten Selektion inkludiert werden. Das Prinzip der
Abschätzung dieser Untergrundbeiträge mittels Simulationen, beruht auf der
Erstellung von Monte-Carlos für jeden relevanten, d.h. möglicherweise
beitragenden, Prozess und der anschließenden Anwendung der selben
Selektionsschnitte, analog zur Selektion in realen Daten. Die aus den
verbleibenden Ereignissen gewonnenen Verteilungen\footnote{Selbstverständlich
ist hierbei auch die mit Gleichung (\ref{eq:mc_scaling}) beschriebene
Skalierung auf Luminosität erforderlich} quantifizieren sodann den Beitrag des
jeweilig betrachteten Prozesses. Alle hier nachstehend aufgeführten Monte-Carlo
Simulationen sind mit ihren Charakteristika bereits in Kapitel
\ref{used_mc_samples} eingeführt worden und werden hier vom motivierenden
Standpunkt der Untergrundabschätzung aus betrachtet.

Mit den größten Beitrag zum Untergrund stellt die Produktion und der Zerfall
eines Top-Antitop Quarkpaares dar. Topquarks haben ein Verzweigungsverhältnis
von beinahe 100\% für den Zerfall in ein $b$-Quark und ein $W$-Boson. Solche
Ereignisse passieren die Selektion, wenn entweder beide $W$-Bosonen leptonisch
in jeweils ein Elektron und ein Antielektronneutrino zerfallen, oder
Jets aus den $b$-Quarks bzw. dem hadronischen Zefall eines oder beider
$W$-Bosonen fälschlicherweise als Elektron rekonstruiert werden. Ähnlich,
jedoch in sehr viel geringerem Umfang, trägt auch die Produktion eines
einzelnen Top-Quarks und dessen konsekutiver Fall zum Untergrund bei. Die
relevanten Zerfallskanäle sind dieselben, wie bei der Paarproduktion bereits
genannt, jedoch ist nun folgerichtig stets ein irrtümlich identifizierter Jet
beteiligt.

In obiger Erläuterung wurde bereits der leptonische Zerfall eines $W$-Bosons in 
ein Elektron und ein Antielektronneutrino angeführt. Dieser Prozess liefert
auch für sich allein gesehen einen Untergrundbeitrag. Hierbei ist ebenso stets
die Beteiligung falsch rekonstruierter Jets implizit. Neben dem Zerfall in ein
Elektron kann auch der verwandte Prozess des Zerfalls in ein $\tau$ und das
entsprechende Antineutrino beisteuern, da $\tau$s eine Verzweigungsverhältnis
von rund $18\%$ (\cite{PhysRevD.86.010001}) für den weiteren Zerfall in ein
Elektron aufweisen.

Ein weitere offensichtlich beitragender Prozess ist die Produktion von $WW$-,
$WZ$- und $ZZ$-Bosonpaaren.

% Erklärung der Plot-Bezeichnungen (ttbar, Singletop...)



\subsection{Datengestützte Untergrundabschätzung}
\label{afb:multijet}

% + MultiJet-Untergrund Bestimmung







%______________________________________________________________________________
%                                                         Asymmetrie Verteilung
\section{Vorwärts-Rückwärts Asymmetrie Verteilung}
\label{afb:afb}

% + Rohe Verteilung
% + Untergund reduzierte Verteilung
% + Vergleich mit Signal-Simulation



%______________________________________________________________________________
%                          Extraktion des effektiven Schwachen Mischungswinkels
\section{Extraktion des effektiven Schwachen Mischungswinkels}
\label{afb:sin2theta}

% + Extraktionsmethode
%   - Template Erzeugung
%   - Faltung mit Detektoreffekten
% + Extraktion
%   - Template Fits
%   - Closure-Test
%   - Resultate (einzeln und kombiniert)
% + Diskussion
%   - Systematische Betrachtungen
%   - Vergleich mit anderen Experimenten
%   - Ausblick
 


