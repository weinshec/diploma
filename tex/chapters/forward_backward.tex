%%%%%%%%%%%%%%%%%%%%%%%%%%%%%%%%%%%%%%%%%%%%%%%%%%%%%%%%%%%%%%%%%%%%%%%%%%%%%%%
%%                                                   FORWARD BACKWARD ASYMMETRY
%%%%%%%%%%%%%%%%%%%%%%%%%%%%%%%%%%%%%%%%%%%%%%%%%%%%%%%%%%%%%%%%%%%%%%%%%%%%%%%
%%                       the chapter about the AFB and the sin2theta extraction



%______________________________________________________________________________
%                     Messung der Asymmetrie und des schwachen Mischungswinkels
\chapter{Messung der Asymmetrie und des schwachen Mischungswinkels}
\label{afb}

\begin{quote}
    The abstact comes last
\end{quote}



%______________________________________________________________________________
%                                Untergrundabschätzung und Kontrollverteilungen
\section{Untergrundabschätzung und Kontrollverteilungen}
\label{afb:kinematics}

% + benutzter Datensatz
% + Selektion / Akzeptanz

% + Benutzte Monte-Carlos
% + Benutzte Gewichte

% + MultiJet-Untergrund Bestimmung

% + Kontroll-Plots der Masse / Winkelverteilung / Phi

Die Messung der Vorwärts-Rückwärts Asymmetrie und die anschließende Extraktion
des schwachen Mischungswinkels basiert, wie schon die vorangegangene
Energiekalibration, auf dem vollen Datensatz des Jahres 2012 mit $8\TeV$
Schwerpunktsenergie und einer integrierten Luminosität\footnote{nach Anwendung
der \ac{GRL}} von $20.3\fb^{-1}$. Zur Selektion der interessanten Ereignisse
werden die in Abschnitt \ref{data_sim_selection:selection} beschriebenen
Schnitte angewendet (siehe Tabelle \ref{tab:data_selection}), wobei die
Unterteilung der Ereignismenge nach solchen mit Beteiligung von
Vorwärts-Elektronen (CF) und reinen Zentral-Zentral Ereignissen (CC) zunächst
beibehalten wird. Die Analyse und Messung wird dann in diesen beiden Kanälen
zuerst getrennt durchgeführt und abschließend zu einem kombiniertem Ergebnis
zusammengefasst.

\begin{table} [h]
    \centering
    \begin{tabular}{|l|r|r|}
        \hline
        & \multicolumn{2}{|c|}{Elektronkandidaten}   \\
        \textbf{Schnitt} & \textbf{CC} & \textbf{CF} \\  % escale CF
        \hline\hline
        \ac{GRL}           & 389741202 & 389741202   \\  % 3954702112
        Detektor Status    & 389740981 & 389740981   \\  % 3945153030
        Trigger            & 158129884 & 158129884   \\  % 1402546381
        primärer Vertex    & 157901943 & 157901943   \\  % 1400899527
        Pseudorapidität    & 149728397 & 122396071   \\  % 1328016208
        Transversal-Impuls &  63811039 &   8342860   \\  %  498956943
        Autor              &  17186000 &   7841721   \\  %  378463236
        ID                 &   4326530 &    995073   \\  %   76603185
        Ladung             &   4266425 &  (995073)   \\  %           
        OQ                 &   4245028 &    988991   \\  %   75178238
        \hline
    \end{tabular}
    \caption[Anzahl der Elektronkandidaten in Daten für CC und CF Selektion]
        {Anzahl der Elektronkandidaten in Daten nach für CC und CF Selektion.
        Die Auflistung ist inklusiv, d.h. untere Schnitte enthalten implizit
        alle vorangegangenen Schnitte.}
    \label{tab:data_selection}
\end{table}

Zur simulationsseitigen Beschreibung des Signalprozesses
$pp \rightarrow \gamma^*/Z(e^+e^-) + X$, also dem elektroschwachen Zerfall
eines $Z$-Bosons bzw. virtuellen Photons in ein Elektronpaar\footnote{auch hier
wird der Begriff \textit{Elektron} wieder synonym für Teilchen und Antiteilchen
verwendet}, wird das von \textsc{Powheg+Pythia} generierte Drell-Yan
Monte-Carlo verwendet (siehe Kapitel \ref{used_mc_samples}).



%______________________________________________________________________________
%                                                         Asymmetrie Verteilung
\section{Vorwärts-Rückwärts Asymmetrie Verteilung}
\label{afb:afb}

% + Rohe Verteilung
% + Untergund reduzierte Verteilung
% + Vergleich mit Signal-Simulation



%______________________________________________________________________________
%                          Extraktion des effektiven Schwachen Mischungswinkels
\section{Extraktion des effektiven Schwachen Mischungswinkels}
\label{afb:sin2theta}

% + Extraktionsmethode
%   - Template Erzeugung
%   - Faltung mit Detektoreffekten
% + Extraktion
%   - Template Fits
%   - Closure-Test
%   - Resultate (einzeln und kombiniert)
% + Diskussion
%   - Systematische Betrachtungen
%   - Vergleich mit anderen Experimenten
%   - Ausblick
 


