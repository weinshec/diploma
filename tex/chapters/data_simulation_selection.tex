%%%%%%%%%%%%%%%%%%%%%%%%%%%%%%%%%%%%%%%%%%%%%%%%%%%%%%%%%%%%%%%%%%%%%%%%%%%%%%%
%%                                                    DATA SIMULATION SELECTION
%%%%%%%%%%%%%%%%%%%%%%%%%%%%%%%%%%%%%%%%%%%%%%%%%%%%%%%%%%%%%%%%%%%%%%%%%%%%%%%
%%                                    about the samples, MC Generation and cuts 



%______________________________________________________________________________
%                                                Daten Simulation und Selektion 
\chapter{Daten, Simulation \& Selektion}

\begin{quote}
    The abstract come last
\end{quote}



%______________________________________________________________________________
%                                                                         Daten 
\section{Daten}
\label{data_sim_selection:data}

\begin{itemize}
    \item Daten von 2012
    \item Parameter: Energie, Bunchespacing, Perioden
    \item Skimms und Slimms, D3PD (auch Verweis auf DAQ)
    \item GRL, Fills/Runs, LumiBlocks
    \item HV-Problem Anfang des Jahres
\end{itemize}

Die Daten, die dieser Arbeit zugrunde liegen, wurden vom ATLAS-Detektor im Jahr
2012 bei erstmalig $8 \TeV$ Schwerpunktsenergie aufgenommen.




%______________________________________________________________________________
%                                                                    Simulation
\section{Simulation}
\label{data_sim_selection:simulation}

\begin{itemize}
    \item Motivation für Simulation (Theorie-Vorhersage)
    \item Mehrschrittiges Prinzip
    \item Eventgeneration (Matrixelemente, Fragmentation,...) (+ Ausgabeformat)
    \item Detektorsimulation (+ Ausgabeformat)
    \item Nötige Korrekturen (Pileup, Effizienzen, Smearing, kFaktors)
    \item Übersicht und kurze(!) Statements zu Samples
    \item LumiScaling
\end{itemize}

\subsection{Erzeugung simulierter Datensätze}
\subsection{Korrekturen}
\subsection{Benutzte simulierte Datensätze}



%______________________________________________________________________________
%                                                                     Selektion 
\section{Selektion}
\label{data_sim_selection:selection}

\begin{itemize}
    \item Motivation für Schnitte (Untergrund-Diskriminierung)
    \item Event-basierte Schnitte (Trigger, Detektor, primVertex)
    \item Elektron-basierte Schnitte (CC/CF, pT, ID, Autor, IQ)
    \item kurze erwähnung andere schnitte (MET,Jets,...)
\end{itemize}


