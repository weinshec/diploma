%%%%%%%%%%%%%%%%%%%%%%%%%%%%%%%%%%%%%%%%%%%%%%%%%%%%%%%%%%%%%%%%%%%%%%%%%%%%%%%
%%                                                                       THEORY
%%%%%%%%%%%%%%%%%%%%%%%%%%%%%%%%%%%%%%%%%%%%%%%%%%%%%%%%%%%%%%%%%%%%%%%%%%%%%%%
%%                                          the standard modell and other stuff



%______________________________________________________________________________
%                                                     Theoretischer Hintergrund
\chapter{Theoretischer Hintergrund}

\begin{quote}
    The abstact comes last
\end{quote}



%______________________________________________________________________________
%                                                               Standard-Modell
\section{Das Standardmodell der Teilchenphysik}
\label{theory:standardmodell}

\begin{itemize}
    \item Einführung (Quarks, Leptonen, Wechselwirkungen, Eichinvarianz)
    \item Starke Wechselwirkung (Farbladung, Confinement)
    \item Elektroschwache Theorie (GSW, Mischungswinkel)
    \item Spontane-Symmetriebrechung (Higgs-Mechanismus)
    \item Hadron-Collider (PDFs, QCD, inelastische Streuung)
\end{itemize}

Das Standardmodell der Teilchenphysik beschreibt die Dynamik subatomarer
Teilchen auf der Basis von drei fundamentalen Wechselwirkungen, der
elektromagnetischen, der schwachen und der starken Wechselwirkung.
Außerordentlich erfolgreich in der Erklärung einer Vielzahl experimenteller
Resultate, wurde die Existenz mehrerer heute bekannter Teilchen bereits vor
deren Entdeckung vom Standardmodell vorhergesagt. Der erst kürzlich erfolgte
experimentelle Nachweis des Higgs-Bosons vervollständigte die Liste der vom
Standardmodell postulierten Teilchen.

Man unterscheidet zwischen zwei Familien von Teilchen, den Bosonen mit
ganzzahligem Spin und den Fermionen mit halbzahligem Spin.

%______________________________________________________________________________
%                                                 Vorwärts-Rückwärts Asymmetrie
\section{Vorwärts-Rückwärts Asymmetrie}
\label{theory:afb}

\begin{itemize}
    \item Theoretische Beschreibung
    \item Messung mit Hadron-Collidern
    \item vorangegangene Messungen / andere Messmethoden
    \item Grund der Messung (Higgs-Constraints)
\end{itemize}



