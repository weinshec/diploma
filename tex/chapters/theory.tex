%%%%%%%%%%%%%%%%%%%%%%%%%%%%%%%%%%%%%%%%%%%%%%%%%%%%%%%%%%%%%%%%%%%%%%%%%%%%%%%
%%                                                                       THEORY
%%%%%%%%%%%%%%%%%%%%%%%%%%%%%%%%%%%%%%%%%%%%%%%%%%%%%%%%%%%%%%%%%%%%%%%%%%%%%%%
%%                                          the standard modell and other stuff



%______________________________________________________________________________
%                                                     Theoretischer Hintergrund
\chapter{Theoretischer Hintergrund}

\begin{quote}
    The abstact comes last
\end{quote}



%______________________________________________________________________________
%                                                               Standard-Modell
\section{Das Standardmodell der Teilchenphysik}
\label{theory:standardmodell}

\begin{itemize}
    \item \sout{Einführung (Quarks, Leptonen, Wechselwirkungen, Eichinvarianz)}
    \item Starke Wechselwirkung (Farbladung, Confinement)
    \item Elektroschwache Theorie (GSW, Mischungswinkel)
    \item Spontane-Symmetriebrechung (Higgs-Mechanismus)
\end{itemize}

Das Standardmodell der Teilchenphysik beschreibt die Dynamik subatomarer
Teilchen auf der Basis von drei fundamentalen Wechselwirkungen, der
elektromagnetischen, der schwachen und der starken Wechselwirkung.
Außerordentlich erfolgreich in der Erklärung einer Vielzahl experimenteller
Resultate, wurde die Existenz mehrerer heute bekannter Teilchen bereits vor
deren Entdeckung vom Standardmodell vorhergesagt. Der erst kürzlich erfolgte
experimentelle Nachweis des Higgs-Bosons durch das ATLAS-Experiment
(\cite{Aad:2012tfa}) komplettiert die Liste der vom Standardmodell postulierten
Teilchen und führt dessen Erfolg fort.

Das Standardmodell unterscheidet zwei Klassen von Teilchen. Die Bosonen mit
ganzzahligem Spin, welche die Austauschteilchen der fundamentalen
Wechselwirkungen repräsentieren und die Fermionen mit halbzahligem Spin, die
die Bausteine der bekannten Materie sind und auschließlich über den Austauch
von Bosonen miteinander wechselwirken.  

Mathematisch beschrieben wird das Standardmodell durch Quantenfeldtheorien. Die
\ac{QED} erklärt dabei die elektromagnetische und die schwache Wechselwirkung,
während die \ac{QCD} die starke Wechselwirkung begründet. Beide sind
Eichtheorien, sodass die Forderung nach lokaler Eichinvarianz der
Lagrangedichten die Existenz der Austauschteilchen, den Eichbosonen, notwendig
macht. Es hängt dabei von der jeweiligen Symmetriegruppe ab, wie viele
Eichbosonen benötigt werden. Das bekannteste Eichboson ist das Photon
($\gamma$), welches die elektromagnetische Wechselwirkung vermittelt, selbst
aber ungeladen und masselos ist. Die schwache Wechselwirkung hingegen kennt
drei Austauschteilchen, die beiden geladenen $W^\pm$ Bosonen und das neutrale
Z-Boson. Alle drei besitzen eine nicht-verschwindende Masse, was nachträgliche
Einführung des Higgs-Mechanismus\footnote{siehe Abschnitt \ref{}}
notwendig gemacht hat. Die Bosonen der starken Wechselwirkung sind die acht
Gluonen ($g$), die wiederum masselos sind keine elektrische Ladung Tragen. Eine
Übersicht über die Austauschteilchen ist in Tabelle \ref{tab:bosons} zu finden.

\begin{table}
    \centering
    \begin{tabular}{|c|c|c|c|c|}
        \hline
        \bf{Eichboson} & \bf{Symbol} & \bf{Masse} $[\GeV]$ & \bf{Ladung} &
        \bf{Wechselwirkung} \\
        \hline\hline
        Photon        & $\gamma$ & $0$       & $0$        & elektromagnetisch \\
        $W^\pm$-Boson & $W^\pm$  & $80,385 $ & $\pm 1\;e$ & schwach \\
        $Z$-Boson     & $Z$      & $91,1876$ & $0$        & schwach \\
        Gluon         & $g$      & $0$       & $0$        & stark \\
        \hline
    \end{tabular}
    \caption[Die Eichbosonen des Standardmodells]
        {Die Eichbosonen des Standardmodells mit Massen
        (\cite{PhysRevD.86.010001}) und Ladung}
    \label{tab:bosons}
\end{table}

Die Fermionen werden unterschieden in solche mit ganzahliger elektrischer
Ladung, den Leptonen, und jene mit drittelzahliger Ladungen, den Quarks. Die
Familie der Leptonen umfasst drei geladene ($Q=-1e$) Teilchen, das Elektron
($e^-$), das Myon ($\mu^-$) und das Tauon ($\tau^-$) und jeweils dazuhörige
ungeladene Neutrinos ($\nu_e,\nu_\mu,\nu_\tau$). Die geladenen Leptonen
wechselwirken sowohl elektromagnetisch, als auch über die schwache
Wechselwirkung, während die elektrisch neutralen Neutrinos lediglich an der
schwachen Wechselwirkung beteiligt sind. Die Massen der geladenen Leptonen sind
wohl bekannt, während für die Massen der Neutrino bloß obere Grenzen angegeben
werden können. Deren absolute Werte und vor allem deren Hierarchie sind
Gegenstand aktueller Forschung\footnote{siehe zum Beispiel
\cite{Winter:2013ema}}. Unter den Quarks, den Konstituenten hadronischer
Materie, existieren solche mit Ladung $Q=+\tfrac{2}{3}e$, die sogenannten
\textit{up-type} Quarks, und solche mit $Q=-\tfrac{1}{3}e$, die
\textit{down-type} Quarks. Alle Quarks koppeln neben der elektromagnetischen
und schwachen Wechselwirkung, zudem noch über die starke Wechselwirkung.
Zu allen Fermionen existieren entsprechende Antiteilchen, die sich zu ihrem
jeweiligen Teilchenpartner nur durch deren Quantenzahlen unterscheiden. Tabelle
\ref{tab:fermions} listet die Fermionen des Standardmodells, wobei auf die
Aufführung der Antiteilchen der Einfachheit halber verzichtet wurde.

\begin{table}
    \centering
    \begin{tabular}{|l|c|l|c|}
        \hline
        \bf{Lepton} & \bf{Symbol} & \bf{Masse} & \bf{Ladung} \\
        \hline\hline
        Elektron          & $e^-$      & $0,511 \MeV$             & $-1\;e$ \\
        Elektron-Neutrino & $\nu_e$    & $<0,205 \eV$ (95\%C.L.)  & $0$     \\
        \hline
        Myon              & $\mu^-$    & $105,658 \MeV$           & $-1\;e$ \\
        Myon-Neutrino     & $\nu_\mu$  & $<0,19 \MeV$  (90\%C.L.) & $0$     \\
        \hline
        Tauon             & $\tau^-$   & $1776,82 \MeV$           & $-1\;e$ \\
        Tauon-Neutrino    & $\nu_\tau$ & $<18,2 \MeV$  (95\%C.L.) & $0$     \\
        \hline\hline
        \bf{Quark} & \bf{Symbol} & \bf{Masse} & \bf{Ladung} \\
        \hline\hline
        up      & $u$ & $2,3 \MeV$    & $+2/3\;e$ \\
        down    & $d$ & $4,8 \MeV$    & $-1/3\;e$ \\
        \hline
        charm   & $c$ & $1,275 \GeV$  & $+2/3\;e$ \\
        strange & $s$ & $95  \MeV$    & $-1/3\;e$ \\
        \hline
        top     & $t$ & $4,66 \GeV$   & $+2/3\;e$ \\
        bottom  & $b$ & $173,07 \GeV$ & $-1/3\;e$ \\
        \hline
    \end{tabular}
    \caption[Die Fermionen des Standardmodells]
        {Die Fermionen des Standardmodells mit Massen
        (\cite{PhysRevD.86.010001}) und Ladung}
    \label{tab:fermions}
\end{table}


\subsection{Phenomenologie der starken Wechselwirkung}
Die starke Wechselwirkung erklärt die Dynamik zwischen den Quarks und wird
mathematisch durch die $SU(3)$ symmetrische \ac{QCD} beschrieben. Die $SU(3)$
Gruppe wird durch 8 Generatoren erzeugt, deren physikalische Entsprechung die 8
Gluonen, der starten Wechselwirkung sind. Die hier relevante Quantenzahl ist
die sogenannte \textit{Farbladung}, welche die Ausprägungen rot, grün und blau
bzw. antirot, antigrün und antiblau annehmen kann. Die Summe aller drei (Anti-)
Farben ist Null und man von einem \textit{weißen} Zustand\footnote{In Analogie
zur additiven Farbmischung}. (Anti-)Quarks tragen stets eine der (Anti-)Farben,
während die Gluonen immer eine Farb- und eine Antifarbladung tragen. Das
direkte Produkt des Farbtripletts mit dem Antifarbtriplett ergibt ein
Farb-Oktett plus ein farbneutrales Singulett. Da in der Natur allerdings keine
farbneutralen freien Gluonen beobachtet werden, gilt dieser Singulettzustand
als unphysikalisch und wird deshalb verworfen. Die acht Zustände des Oktetts
werden den acht Gluonfeldern zugeordnet. (Anti-)Quarks tragen immer eine
(Anti-)Farbladung und treten in der Natur ebenfalls nicht als freie Teilchen
auf. Dieses Phänomen bezeichnet man als \textit{Confinement} und führt dazu,
dass Quarks nur in Bindungszuständen, sogenannten Hadronen, vorkommen.

\development
Teilchen sein müssen deren Konstituenten in Summe weiß sein müssen. Dies ist
möglich durch die Addition aller drei Farben bzw. Antifarben
(rot+grün+blau=weiß) oder die Gegenseitige Aufhebung einer Farbe mit ihrerer
jeweiligen Antifarbe. Die zusammengesetzten Teilchen sind dann Mesonen
(2Teilchen) oder Baryonen(3Teilchen) die man alle unter dem Begriff Hadronen
zusammenfasst.
Die Tatsache des Confinement spiegelt sich auch in der Kopplungsstärke wieder,
denn entgegen der Elektromagnetischen Wechselwirkung bei der mit größer
werdenden Abständen das Potential immer kleiner wird, wächst die Kraft der
starken WW mit größeren Abständen an. Die Ursache hierfür ist die Asymptotische
Freiheit, die in allen nicht-abelschen SU(3) Theorien, wie der QCD auftritt und
bewirkt, dass mit kleiner werdenden Energien also größeren Abständen die
Kopplungskonstante steigt. siehe Bild.

%______________________________________________________________________________
%                                                               Hadron-Collider
\section{Physik mit Hadron-Collidern}
\label{theory:hadron_collider}

\begin{itemize}
    \item Luminosität, Schwerpunktsenergie
    \item PDFs
\end{itemize}



%______________________________________________________________________________
%                                                 Vorwärts-Rückwärts Asymmetrie
\section{Vorwärts-Rückwärts Asymmetrie}
\label{theory:afb}

\begin{itemize}
    \item Theoretische Beschreibung
    \item Collins-Soper
    \item vorangegangene Messungen / andere Messmethoden
    \item Grund der Messung (Higgs-Constraints)
\end{itemize}



