%%%%%%%%%%%%%%%%%%%%%%%%%%%%%%%%%%%%%%%%%%%%%%%%%%%%%%%%%%%%%%%%%%%%%%%%%%%%%%%
%%                                                           ENERGY CALIBRATION
%%%%%%%%%%%%%%%%%%%%%%%%%%%%%%%%%%%%%%%%%%%%%%%%%%%%%%%%%%%%%%%%%%%%%%%%%%%%%%%
%%                                          forward electron energy calibration 



%______________________________________________________________________________
%                                    Energiekalibration der Vorwärts-Elektronen
%
\chapter{Energiekalibration der Vorwärts-Elektronen}
\label{energy_calibration}

\begin{quote}
    Die Energiekalibration des elektromagnetischen Kalorimeters für Elektronen 
    ist von essentieller Bedeutung für Messung der Vorwärts-Rückwärts-
    Asymmetrie. Dieses Kapitel beschreibt die Kalibration für Elektronen mit
    hohen Pseudorapiditäten. Dabei wird zunächst auf \development 
\end{quote}



%______________________________________________________________________________
%                                                                    Grundlagen
%
\section{Grundlagen}
\label{energy_calibration:grundlagen}
\begin{itemize}
    \item Messung der Energie mit Kalorimeter (Verweis auf Detektor Kapitel)
    \item Notwendigkeit der Energiekalibration (Benötigte Präzision für 
        Vorwärts-Rückwärts-Asymmetrie)
    \item Vorangegangene Schritte ( Calibration-Hits, Test-Beam-Runs, 3-Stufen)
    \item Was im folgenden passiert (Vorwärts-Kalibration, in-situ
        kalibration, was ist noch zu kalibrieren)
\end{itemize}



%______________________________________________________________________________
%                                                      Beschreibung der Methode 
%
\section{Beschreibung der Methode}
\label{energy_calibration:beschreibung_der_methode}

\begin{itemize}
    \item \sout{Betrachteter Prozess, CF-Ereignisse}
    \item Definition der Korrektur-Faktoren
    \item Annahmen (perfekte zentral-Elektronen)
    \item Vereinfachung der Formeln und Extraktion
    \item 2-stufige Exktraktion
    \item Selektion und Samples
    \item Fit-Modelle, Effizienzkurve
\end{itemize}

Für die Kalibration der Vorwärts-Elektronen betrachtet man den elektroschwachen
Zerfalls-Prozess $Z/\gamma^* \rightarrow ee$ eines Z-Bosons in zwei
Elektronen\footnote{Die Bezeichnung \textit{Elektron} wird hier und im
Folgenden synonym für Elektronen und Positronen verwendet}. Dabei werden nur
Ereignisse selektiert, in denen eines der beiden Elektronen im Zentral-Bereich,
das andere im Vorwärts-Bereich detektiert wird. Diese Einbeziehung der Zentral-
Elektronen ist aus mehreren Gründen notwendig. Zum einen stehen entsprechende
Elektron-Trigger\footnote{siehe hierzu Kapitel
\ref{experimenteller_aufbau:atlas_detector:trigger-system}}
nur im Zentral-Bereich zur Verfügung. Zum anderen wäre die Selektion von
Ereignissen mit beiden Elektronen im Vorwärts-Bereich, in hohem Maße von
Untergrund-Prozessen dominiert, bei denen andere Objekte, überwiegend Jets,
fälschlicherweise als Elektronen rekonstruiert werden\footnote{vgl. Kapitel
\ref{experimenteller_aufbau:elektronen_in_atlas}}.
Allerdings stellt die Hinzunahme der Zentral-Elektronen die Kalibration der
Vorwärts-Elektronen ab initio in die Abhängigkeit einer vorangegangen 
Kalibration der Zentral-Elektronen.

Im folgenden Abschnitt werden zunächst einige grundlegende Annahmen eingeführt
und die Kalibration-Konstanten für Vorwärts-Elektronen definiert.


\subsection{Definitionen und Annahmen}
\label{energy_calibration:beschreibung_der_methode:definitionen_und_annahmen}

Die invariante Masse zweier relativistischer Teilchen, deren Ruhemassen
gegenüber ihren Energien vernachlässigt werden können, ergibt sich aus
\begin{equation}
    \label{invariant_mass:basic}
    m = \sqrt{ 2 \cdot E_1 E_2 (1-\cos\theta_{12}) }
\end{equation}
Dabei bezeichnet $E_i$ die Energie des Teilchens $i$ und $\theta_{12}$ den
Öffnungswinkel zwischen beiden. Für die hier beschriebene Kalibration der
Vorwärts-Elektronen identifiziert man o.B.d.A. die Energie des
Zentral-Elektrons mit $E_1$ und die Energie des Vorwärts-Elektrons mit $E_2$.

Der Kalibrations-Faktor $\alpha$ wird definiert, um die Abweichung zwischen
wahrer und gemessener Energie zu korrigieren, wobei für die wahre Energie
der Wert aus \acs{MC}-Simulationen angenommen wird.
\development
\begin{equation}
    \label{definition:energy_scale}
    E_\text{(meas)} = E_\text{(true)} (1+\alpha)
\end{equation}
Die Indices (meas) und (true) zeigen dabei jeweils die gemessene bzw. wahre
Größe an. 

Für die Messung der invarianten Masse heißt das:
\begin{equation}
    \begin{array}{rcl}
        m_\text{(meas)} &=& \sqrt{ 2 \cdot E_{1,\text{(meas)}} E_{2,\text{(meas)}}
                            (1-\cos\theta_{12})} \\
                        &=& \sqrt{ 2 \cdot E_{1,\text{(true)}} E_{2,\text{(true)}}
                            (1+\alpha_1)(1+\alpha_2)(1-\cos\theta_{12})}

    \end{array}
\end{equation}


%______________________________________________________________________________
%                                       Exktraktion der Kalibrations-Konstanten
%
\section{Extraktion der Kalibrations-Konstanten}
\label{energy_calibration:extraktion_der_kalibrations-konstanten}
\begin{itemize}
    \item Beispielhafte Fits
    \item Exktraktion der Energy-Scales
    \item Extraktion der Constant-Terms
    \item Systematiken
\end{itemize}



%______________________________________________________________________________
%                                                       Ergebnisse und Ausblick
%
\section{Ergebnisse und Ausblick}
\label{energy_calibration:ergebnisse_und_ausblick}
\begin{itemize}
    \item Zusammenfassung
    \item Diskussion ( Dominiert von Zentral-Skalen, Modell-Probleme )
    \item Verbesserungen ( Neue Zentral-Skalen, Template-Ansatz )
\end{itemize}


