%%%%%%%%%%%%%%%%%%%%%%%%%%%%%%%%%%%%%%%%%%%%%%%%%%%%%%%%%%%%%%%%%%%%%%%%%%%%%%%
%%                                                    APPENDIX: PYTHIA_TEMPLATE
%%%%%%%%%%%%%%%%%%%%%%%%%%%%%%%%%%%%%%%%%%%%%%%%%%%%%%%%%%%%%%%%%%%%%%%%%%%%%%%
%%                                the pythia joboptions for template generation 



%______________________________________________________________________________
%                                                     PYTHIA8 Parameter
\chapter{\textsc{PYTHIA8} Konfigurationen}

Die gezeigten Ausschnitte sind C++ Quellcode und werden verwendet, um
\textsc{Pythia8} für die jeweils angegeben Zwecke zu konfigurieren. Weitere
Details zu den verwendeten Optionen sind im Manual zu finden.

\section{Studie zur Dilution}
Die Variabeln \texttt{mMax} und \texttt{mMin} werden zuvor extern entsprechend
den Anforderungen gesetzt. \texttt{mySeed} wird bei jedem Aufruf aus
elektrischem Rauschen bestimmt.

\lstset{language=C++}
\begin{small}
\begin{lstlisting}[frame=single]
// Set C.M. energy to 8 TeV
pythia.readString("Beams:eCM=8000.");

// Use custom random seed
pythia.readString("Random:SetSeed=on");
pythia.readString("Random:seed=mySeed");

// Choose process: qqbar->gamma*/Z
pythia.readString("WeakSingleBoson:ffbar2gmZ=on");

// Set decay mode to electrons only
pythia.readString("23:onMode=off");
pythia.readString("23:onIfAny=11");

// Specify invariant mass range
pythia.readString("PhaseSpace:mHatMax=mMax");
pythia.readString("PhaseSpace:mHatMin=mMin");
\end{lstlisting}
\end{small}
